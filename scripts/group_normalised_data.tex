% Options for packages loaded elsewhere
\PassOptionsToPackage{unicode}{hyperref}
\PassOptionsToPackage{hyphens}{url}
\PassOptionsToPackage{dvipsnames,svgnames,x11names}{xcolor}
%
\documentclass[
  letterpaper,
  DIV=11,
  numbers=noendperiod]{scrartcl}

\usepackage{amsmath,amssymb}
\usepackage{iftex}
\ifPDFTeX
  \usepackage[T1]{fontenc}
  \usepackage[utf8]{inputenc}
  \usepackage{textcomp} % provide euro and other symbols
\else % if luatex or xetex
  \usepackage{unicode-math}
  \defaultfontfeatures{Scale=MatchLowercase}
  \defaultfontfeatures[\rmfamily]{Ligatures=TeX,Scale=1}
\fi
\usepackage{lmodern}
\ifPDFTeX\else  
    % xetex/luatex font selection
\fi
% Use upquote if available, for straight quotes in verbatim environments
\IfFileExists{upquote.sty}{\usepackage{upquote}}{}
\IfFileExists{microtype.sty}{% use microtype if available
  \usepackage[]{microtype}
  \UseMicrotypeSet[protrusion]{basicmath} % disable protrusion for tt fonts
}{}
\makeatletter
\@ifundefined{KOMAClassName}{% if non-KOMA class
  \IfFileExists{parskip.sty}{%
    \usepackage{parskip}
  }{% else
    \setlength{\parindent}{0pt}
    \setlength{\parskip}{6pt plus 2pt minus 1pt}}
}{% if KOMA class
  \KOMAoptions{parskip=half}}
\makeatother
\usepackage{xcolor}
\setlength{\emergencystretch}{3em} % prevent overfull lines
\setcounter{secnumdepth}{-\maxdimen} % remove section numbering
% Make \paragraph and \subparagraph free-standing
\makeatletter
\ifx\paragraph\undefined\else
  \let\oldparagraph\paragraph
  \renewcommand{\paragraph}{
    \@ifstar
      \xxxParagraphStar
      \xxxParagraphNoStar
  }
  \newcommand{\xxxParagraphStar}[1]{\oldparagraph*{#1}\mbox{}}
  \newcommand{\xxxParagraphNoStar}[1]{\oldparagraph{#1}\mbox{}}
\fi
\ifx\subparagraph\undefined\else
  \let\oldsubparagraph\subparagraph
  \renewcommand{\subparagraph}{
    \@ifstar
      \xxxSubParagraphStar
      \xxxSubParagraphNoStar
  }
  \newcommand{\xxxSubParagraphStar}[1]{\oldsubparagraph*{#1}\mbox{}}
  \newcommand{\xxxSubParagraphNoStar}[1]{\oldsubparagraph{#1}\mbox{}}
\fi
\makeatother

\usepackage{color}
\usepackage{fancyvrb}
\newcommand{\VerbBar}{|}
\newcommand{\VERB}{\Verb[commandchars=\\\{\}]}
\DefineVerbatimEnvironment{Highlighting}{Verbatim}{commandchars=\\\{\}}
% Add ',fontsize=\small' for more characters per line
\usepackage{framed}
\definecolor{shadecolor}{RGB}{241,243,245}
\newenvironment{Shaded}{\begin{snugshade}}{\end{snugshade}}
\newcommand{\AlertTok}[1]{\textcolor[rgb]{0.68,0.00,0.00}{#1}}
\newcommand{\AnnotationTok}[1]{\textcolor[rgb]{0.37,0.37,0.37}{#1}}
\newcommand{\AttributeTok}[1]{\textcolor[rgb]{0.40,0.45,0.13}{#1}}
\newcommand{\BaseNTok}[1]{\textcolor[rgb]{0.68,0.00,0.00}{#1}}
\newcommand{\BuiltInTok}[1]{\textcolor[rgb]{0.00,0.23,0.31}{#1}}
\newcommand{\CharTok}[1]{\textcolor[rgb]{0.13,0.47,0.30}{#1}}
\newcommand{\CommentTok}[1]{\textcolor[rgb]{0.37,0.37,0.37}{#1}}
\newcommand{\CommentVarTok}[1]{\textcolor[rgb]{0.37,0.37,0.37}{\textit{#1}}}
\newcommand{\ConstantTok}[1]{\textcolor[rgb]{0.56,0.35,0.01}{#1}}
\newcommand{\ControlFlowTok}[1]{\textcolor[rgb]{0.00,0.23,0.31}{\textbf{#1}}}
\newcommand{\DataTypeTok}[1]{\textcolor[rgb]{0.68,0.00,0.00}{#1}}
\newcommand{\DecValTok}[1]{\textcolor[rgb]{0.68,0.00,0.00}{#1}}
\newcommand{\DocumentationTok}[1]{\textcolor[rgb]{0.37,0.37,0.37}{\textit{#1}}}
\newcommand{\ErrorTok}[1]{\textcolor[rgb]{0.68,0.00,0.00}{#1}}
\newcommand{\ExtensionTok}[1]{\textcolor[rgb]{0.00,0.23,0.31}{#1}}
\newcommand{\FloatTok}[1]{\textcolor[rgb]{0.68,0.00,0.00}{#1}}
\newcommand{\FunctionTok}[1]{\textcolor[rgb]{0.28,0.35,0.67}{#1}}
\newcommand{\ImportTok}[1]{\textcolor[rgb]{0.00,0.46,0.62}{#1}}
\newcommand{\InformationTok}[1]{\textcolor[rgb]{0.37,0.37,0.37}{#1}}
\newcommand{\KeywordTok}[1]{\textcolor[rgb]{0.00,0.23,0.31}{\textbf{#1}}}
\newcommand{\NormalTok}[1]{\textcolor[rgb]{0.00,0.23,0.31}{#1}}
\newcommand{\OperatorTok}[1]{\textcolor[rgb]{0.37,0.37,0.37}{#1}}
\newcommand{\OtherTok}[1]{\textcolor[rgb]{0.00,0.23,0.31}{#1}}
\newcommand{\PreprocessorTok}[1]{\textcolor[rgb]{0.68,0.00,0.00}{#1}}
\newcommand{\RegionMarkerTok}[1]{\textcolor[rgb]{0.00,0.23,0.31}{#1}}
\newcommand{\SpecialCharTok}[1]{\textcolor[rgb]{0.37,0.37,0.37}{#1}}
\newcommand{\SpecialStringTok}[1]{\textcolor[rgb]{0.13,0.47,0.30}{#1}}
\newcommand{\StringTok}[1]{\textcolor[rgb]{0.13,0.47,0.30}{#1}}
\newcommand{\VariableTok}[1]{\textcolor[rgb]{0.07,0.07,0.07}{#1}}
\newcommand{\VerbatimStringTok}[1]{\textcolor[rgb]{0.13,0.47,0.30}{#1}}
\newcommand{\WarningTok}[1]{\textcolor[rgb]{0.37,0.37,0.37}{\textit{#1}}}

\providecommand{\tightlist}{%
  \setlength{\itemsep}{0pt}\setlength{\parskip}{0pt}}\usepackage{longtable,booktabs,array}
\usepackage{calc} % for calculating minipage widths
% Correct order of tables after \paragraph or \subparagraph
\usepackage{etoolbox}
\makeatletter
\patchcmd\longtable{\par}{\if@noskipsec\mbox{}\fi\par}{}{}
\makeatother
% Allow footnotes in longtable head/foot
\IfFileExists{footnotehyper.sty}{\usepackage{footnotehyper}}{\usepackage{footnote}}
\makesavenoteenv{longtable}
\usepackage{graphicx}
\makeatletter
\def\maxwidth{\ifdim\Gin@nat@width>\linewidth\linewidth\else\Gin@nat@width\fi}
\def\maxheight{\ifdim\Gin@nat@height>\textheight\textheight\else\Gin@nat@height\fi}
\makeatother
% Scale images if necessary, so that they will not overflow the page
% margins by default, and it is still possible to overwrite the defaults
% using explicit options in \includegraphics[width, height, ...]{}
\setkeys{Gin}{width=\maxwidth,height=\maxheight,keepaspectratio}
% Set default figure placement to htbp
\makeatletter
\def\fps@figure{htbp}
\makeatother

\usepackage{listings}
\lstset{
  basicstyle=\ttfamily\small,  % Font style
  breaklines=true,             % Enable line wrapping
  breakatwhitespace=true,      % Break lines at whitespaces
  frame=single,                % Add a frame around code blocks
  prebreak=\raisebox{0ex}[0ex][0ex]{\ensuremath{\hookleftarrow}}, % Indicate broken lines
  postbreak=\raisebox{0ex}[0ex][0ex]{\ensuremath{\hookrightarrow}}
}
\KOMAoption{captions}{tableheading}
\makeatletter
\@ifpackageloaded{caption}{}{\usepackage{caption}}
\AtBeginDocument{%
\ifdefined\contentsname
  \renewcommand*\contentsname{Table of contents}
\else
  \newcommand\contentsname{Table of contents}
\fi
\ifdefined\listfigurename
  \renewcommand*\listfigurename{List of Figures}
\else
  \newcommand\listfigurename{List of Figures}
\fi
\ifdefined\listtablename
  \renewcommand*\listtablename{List of Tables}
\else
  \newcommand\listtablename{List of Tables}
\fi
\ifdefined\figurename
  \renewcommand*\figurename{Figure}
\else
  \newcommand\figurename{Figure}
\fi
\ifdefined\tablename
  \renewcommand*\tablename{Table}
\else
  \newcommand\tablename{Table}
\fi
}
\@ifpackageloaded{float}{}{\usepackage{float}}
\floatstyle{ruled}
\@ifundefined{c@chapter}{\newfloat{codelisting}{h}{lop}}{\newfloat{codelisting}{h}{lop}[chapter]}
\floatname{codelisting}{Listing}
\newcommand*\listoflistings{\listof{codelisting}{List of Listings}}
\makeatother
\makeatletter
\makeatother
\makeatletter
\@ifpackageloaded{caption}{}{\usepackage{caption}}
\@ifpackageloaded{subcaption}{}{\usepackage{subcaption}}
\makeatother

\ifLuaTeX
  \usepackage{selnolig}  % disable illegal ligatures
\fi
\usepackage{bookmark}

\IfFileExists{xurl.sty}{\usepackage{xurl}}{} % add URL line breaks if available
\urlstyle{same} % disable monospaced font for URLs
\hypersetup{
  pdftitle={Normalisation Quality Control},
  pdfauthor={Payel Sardar},
  colorlinks=true,
  linkcolor={blue},
  filecolor={Maroon},
  citecolor={Blue},
  urlcolor={Blue},
  pdfcreator={LaTeX via pandoc}}


\title{Normalisation Quality Control}
\author{Payel Sardar}
\date{}

\begin{document}
\maketitle


\section{Description}\label{description}

This script takes in individual normalised files and merges then into a
unified dataset. A Box plot is then created to check if the samples for
corresponding carbon source (glucose and cellobiose) get clustered
together.

\subsubsection{Setting the home working
directory.}\label{setting-the-home-working-directory.}

\begin{Shaded}
\begin{Highlighting}[]
\FunctionTok{setwd}\NormalTok{(}\StringTok{"D:/KCL2024/Courses/7BBG1002\_Cloud\_computing/Project"}\NormalTok{)}
\end{Highlighting}
\end{Shaded}

\subsubsection{Loading necessary
packages}\label{loading-necessary-packages}

\begin{Shaded}
\begin{Highlighting}[]
\FunctionTok{library}\NormalTok{(dplyr)}
\end{Highlighting}
\end{Shaded}

\begin{verbatim}

Attaching package: 'dplyr'
\end{verbatim}

\begin{verbatim}
The following objects are masked from 'package:stats':

    filter, lag
\end{verbatim}

\begin{verbatim}
The following objects are masked from 'package:base':

    intersect, setdiff, setequal, union
\end{verbatim}

\begin{Shaded}
\begin{Highlighting}[]
\FunctionTok{library}\NormalTok{(ggplot2)}
\end{Highlighting}
\end{Shaded}

\begin{verbatim}
Warning: package 'ggplot2' was built under R version 4.4.2
\end{verbatim}

\begin{Shaded}
\begin{Highlighting}[]
\FunctionTok{library}\NormalTok{(tidyr)}
\end{Highlighting}
\end{Shaded}

\subsubsection{Reading the file on sample
information}\label{reading-the-file-on-sample-information}

\begin{Shaded}
\begin{Highlighting}[]
\NormalTok{sample\_info }\OtherTok{\textless{}{-}} \FunctionTok{read.csv}\NormalTok{(}\StringTok{"../metadata/sample\_types.csv"}\NormalTok{, }
                        \AttributeTok{sep =} \StringTok{","}\NormalTok{, }\AttributeTok{header =} \ConstantTok{TRUE}\NormalTok{)}
\FunctionTok{head}\NormalTok{(sample\_info)}
\end{Highlighting}
\end{Shaded}

\begin{verbatim}
         Run    BioSample  Bases   Bytes carbon_source Experiment GEO_Accession
1 SRR1166442 SAMN02639514 2.03 G 1.29 Gb       Glucose  SRX468698    GSM1324496
2 SRR1166443 SAMN02639516 2.25 G 1.42 Gb       Glucose  SRX468699    GSM1324497
3 SRR1166444 SAMN02639513 1.74 G 1.05 Gb       Glucose  SRX468700    GSM1324498
4 SRR1166445 SAMN02639515 2.28 G 1.44 Gb    Cellobiose  SRX468701    GSM1324499
5 SRR1166446 SAMN02639512 1.91 G 1.20 Gb    Cellobiose  SRX468702    GSM1324500
6 SRR1166447 SAMN02639517 1.73 G 1.04 Gb    Cellobiose  SRX468703    GSM1324501
          create_date Sample.Name            source_name
1 2014-02-1010:25:00Z  GSM1324496    Glucose-grown cells
2 2014-02-1010:25:00Z  GSM1324497    Glucose-grown cells
3 2014-02-1010:24:00Z  GSM1324498    Glucose-grown cells
4 2014-02-1010:25:00Z  GSM1324499 Cellobiose-grown cells
5 2014-02-1010:24:00Z  GSM1324500 Cellobiose-grown cells
6 2014-02-1010:24:00Z  GSM1324501 Cellobiose-grown cells
\end{verbatim}

Next, the sample\_ids(Run) for each carbon\_source is extracted

\begin{Shaded}
\begin{Highlighting}[]
\NormalTok{glucose\_samples }\OtherTok{\textless{}{-}}\NormalTok{ sample\_info }\SpecialCharTok{\%\textgreater{}\%} 
  \FunctionTok{filter}\NormalTok{(carbon\_source }\SpecialCharTok{==} \StringTok{"Glucose"}\NormalTok{) }\SpecialCharTok{\%\textgreater{}\%} 
  \FunctionTok{pull}\NormalTok{(Run)}
\NormalTok{cellobiose\_samples }\OtherTok{\textless{}{-}}\NormalTok{ sample\_info }\SpecialCharTok{\%\textgreater{}\%} 
  \FunctionTok{filter}\NormalTok{(carbon\_source }\SpecialCharTok{==} \StringTok{"Cellobiose"}\NormalTok{) }\SpecialCharTok{\%\textgreater{}\%} 
  \FunctionTok{pull}\NormalTok{(Run)}
\end{Highlighting}
\end{Shaded}

The function below reads multiple normalized RPKM files (in CSV format)
for a set of sample IDs corresponding to a particular carbon\_source,
extracts the first column (Geneid) and the last column
(normalized\_rpkm) from each file, and merges the data frames by the
Geneid column to create a combined dataset.

\begin{Shaded}
\begin{Highlighting}[]
\NormalTok{group\_samples }\OtherTok{\textless{}{-}} \ControlFlowTok{function}\NormalTok{(sample\_ids, }\AttributeTok{prefix =} \StringTok{"normalized\_"}\NormalTok{, }
                          \AttributeTok{folder =} \StringTok{"../data/processed/Normalized\_data/"}\NormalTok{) \{}
  
  \CommentTok{\# Initializing an empty list to store data}
\NormalTok{  data\_list }\OtherTok{\textless{}{-}} \FunctionTok{list}\NormalTok{()}
  
  \ControlFlowTok{for}\NormalTok{ (sample\_id }\ControlFlowTok{in}\NormalTok{ sample\_ids) \{}
    
\NormalTok{    file\_name }\OtherTok{\textless{}{-}} \FunctionTok{paste0}\NormalTok{(folder, prefix, sample\_id, }\StringTok{".csv"}\NormalTok{)}
    
    \CommentTok{\# Reading the normalized file}
\NormalTok{    data }\OtherTok{\textless{}{-}} \FunctionTok{read.csv}\NormalTok{(file\_name)}
    
    \CommentTok{\# Extracting the first column(Geneid) and the last column(normalized\_rpkm)}
\NormalTok{    data\_subset }\OtherTok{\textless{}{-}}\NormalTok{ data[, }\FunctionTok{c}\NormalTok{(}\DecValTok{1}\NormalTok{, }\FunctionTok{ncol}\NormalTok{(data))]  }
    
    \CommentTok{\# Renaming the last column to the sample ID for identification}
    \FunctionTok{colnames}\NormalTok{(data\_subset)[}\DecValTok{2}\NormalTok{] }\OtherTok{\textless{}{-}}\NormalTok{ sample\_id}
    
    \CommentTok{\# Appending the file data the list}
\NormalTok{    data\_list[[sample\_id]] }\OtherTok{\textless{}{-}}\NormalTok{ data\_subset}
\NormalTok{  \}}
  
  \CommentTok{\# Merging all data frames by "gene\_id"}
\NormalTok{  merged\_data }\OtherTok{\textless{}{-}} \FunctionTok{Reduce}\NormalTok{(}\ControlFlowTok{function}\NormalTok{(x, y) }\FunctionTok{full\_join}\NormalTok{(x, y, }\AttributeTok{by =} \StringTok{"Geneid"}\NormalTok{), data\_list)}
  
  \FunctionTok{return}\NormalTok{(merged\_data)}
\NormalTok{\}}
\end{Highlighting}
\end{Shaded}

Applying the function for glucose and cellobiose Samples

\begin{Shaded}
\begin{Highlighting}[]
\NormalTok{glucose\_data }\OtherTok{\textless{}{-}} \FunctionTok{group\_samples}\NormalTok{(glucose\_samples)}
\NormalTok{cellobiose\_data }\OtherTok{\textless{}{-}} \FunctionTok{group\_samples}\NormalTok{(cellobiose\_samples)}
\end{Highlighting}
\end{Shaded}

Veiwing the merged datasets

\begin{Shaded}
\begin{Highlighting}[]
\FunctionTok{head}\NormalTok{(glucose\_data)}
\end{Highlighting}
\end{Shaded}

\begin{verbatim}
     Geneid SRR1166442 SRR1166443 SRR1166444
1   YAL068C  8.4379100 11.1595834  10.898455
2 YAL067W-A  0.3630822  0.3290228   0.564276
3   YAL067C 55.5291618 53.1827885  40.971760
4   YAL065C  3.4225421  3.5537864   2.825754
5 YAL064W-B  3.3315888  4.0691744   3.292351
6 YAL064C-A 10.1396180 10.1729361   8.273086
\end{verbatim}

\begin{Shaded}
\begin{Highlighting}[]
\FunctionTok{head}\NormalTok{(cellobiose\_data)}
\end{Highlighting}
\end{Shaded}

\begin{verbatim}
     Geneid SRR1166445 SRR1166446 SRR1166447
1   YAL068C 11.6956377  9.3393236 10.1604814
2 YAL067W-A  0.5115573  0.4956395  0.1336905
3   YAL067C 15.5382521 17.9623442 15.2407221
4   YAL065C  1.8685722  2.3360375  2.5991929
5 YAL064W-B  3.0000621  2.7435795  2.8801365
6 YAL064C-A 10.1022500  7.8599846 10.0004739
\end{verbatim}

Saving the merged dataframes as CSV files in the Normalized\_data folder
for downstream processing

\begin{Shaded}
\begin{Highlighting}[]
\FunctionTok{write.csv}\NormalTok{(glucose\_data, }\StringTok{"../data/processed/Normalized\_data/glucose\_merged.csv"}\NormalTok{, }\AttributeTok{row.names =} \ConstantTok{FALSE}\NormalTok{)}
\FunctionTok{write.csv}\NormalTok{(cellobiose\_data, }\StringTok{"../data/processed/Normalized\_data/cellobiose\_merged.csv"}\NormalTok{, }\AttributeTok{row.names =} \ConstantTok{FALSE}\NormalTok{)}
\end{Highlighting}
\end{Shaded}

Merging the two datasets for visualisation

\begin{Shaded}
\begin{Highlighting}[]
\NormalTok{merged\_norm\_rpkm }\OtherTok{\textless{}{-}} \FunctionTok{merge}\NormalTok{(glucose\_data, cellobiose\_data, }
                          \AttributeTok{by.x =} \StringTok{"Geneid"}\NormalTok{, }\AttributeTok{by.y =} \StringTok{"Geneid"}\NormalTok{)}
\FunctionTok{write.csv}\NormalTok{(merged\_norm\_rpkm, }\StringTok{"../data/processed/Normalized\_data/CombinedNF.csv"}\NormalTok{, }\AttributeTok{row.names =} \ConstantTok{FALSE}\NormalTok{)}
\end{Highlighting}
\end{Shaded}

\begin{Shaded}
\begin{Highlighting}[]
\CommentTok{\# Removing gene\_id column}
\NormalTok{merged\_data\_numeric }\OtherTok{\textless{}{-}}\NormalTok{ merged\_norm\_rpkm[, }\SpecialCharTok{{-}}\DecValTok{1}\NormalTok{]}
\CommentTok{\# Setting gene IDs as row names}
\FunctionTok{row.names}\NormalTok{(merged\_data\_numeric) }\OtherTok{\textless{}{-}}\NormalTok{ merged\_norm\_rpkm}\SpecialCharTok{$}\NormalTok{Geneid  }

\FunctionTok{head}\NormalTok{(merged\_data\_numeric)}
\end{Highlighting}
\end{Shaded}

\begin{verbatim}
      SRR1166442 SRR1166443 SRR1166444 SRR1166445 SRR1166446 SRR1166447
Q0020 0.10040356 0.21229852 0.13653495 0.50926167 0.54823925  0.3327265
Q0045 0.03438535 0.04673969 0.04007942 0.27614533 0.17602152  0.1329409
Q0050 0.11015667 0.09982329 0.07703888 0.21418017 0.28195064  0.1946919
Q0055 0.08606393 0.05849294 0.06269734 0.32739668 0.36346900  0.3446245
Q0060 0.02211077 0.04007329 0.02577222 0.05607455 0.02263738  0.0000000
Q0065 0.00000000 0.02992906 0.00000000 0.05583965 0.03381383  0.0000000
\end{verbatim}

Preparing the Data for making the Box plot

\begin{Shaded}
\begin{Highlighting}[]
\NormalTok{data\_long }\OtherTok{\textless{}{-}} \FunctionTok{pivot\_longer}\NormalTok{(merged\_data\_numeric, }
                          \AttributeTok{cols =} \FunctionTok{everything}\NormalTok{(),  }
                          \AttributeTok{names\_to =} \StringTok{"Sample"}\NormalTok{,   }
                          \AttributeTok{values\_to =} \StringTok{"Norm\_RPKM"}\NormalTok{)}
\NormalTok{data\_long}\SpecialCharTok{$}\NormalTok{log2\_norm\_rpkm }\OtherTok{\textless{}{-}} \FunctionTok{log2}\NormalTok{(data\_long}\SpecialCharTok{$}\NormalTok{Norm\_RPKM }\SpecialCharTok{+} \FloatTok{0.001}\NormalTok{) }\CommentTok{\# to avoid log2(0)}

\NormalTok{sample\_conditions }\OtherTok{\textless{}{-}} \FunctionTok{c}\NormalTok{(}\FunctionTok{rep}\NormalTok{(}\StringTok{"Glucose"}\NormalTok{, }\DecValTok{3}\NormalTok{), }\FunctionTok{rep}\NormalTok{(}\StringTok{"Cellobiose"}\NormalTok{, }\DecValTok{3}\NormalTok{))}

\NormalTok{data\_long}\SpecialCharTok{$}\NormalTok{Condition }\OtherTok{\textless{}{-}} 
  \FunctionTok{rep}\NormalTok{(sample\_conditions, }\AttributeTok{times =} \FunctionTok{nrow}\NormalTok{(data\_long)}\SpecialCharTok{/}\FunctionTok{length}\NormalTok{(sample\_conditions))}

\FunctionTok{head}\NormalTok{(data\_long)}
\end{Highlighting}
\end{Shaded}

\begin{verbatim}
# A tibble: 6 x 4
  Sample     Norm_RPKM log2_norm_rpkm Condition 
  <chr>          <dbl>          <dbl> <chr>     
1 SRR1166442     0.100         -3.30  Glucose   
2 SRR1166443     0.212         -2.23  Glucose   
3 SRR1166444     0.137         -2.86  Glucose   
4 SRR1166445     0.509         -0.971 Cellobiose
5 SRR1166446     0.548         -0.864 Cellobiose
6 SRR1166447     0.333         -1.58  Cellobiose
\end{verbatim}

\subsubsection{Creating the box plots}\label{creating-the-box-plots}

\begin{Shaded}
\begin{Highlighting}[]
\NormalTok{plot }\OtherTok{\textless{}{-}} \FunctionTok{ggplot}\NormalTok{(data\_long, }\FunctionTok{aes}\NormalTok{(}\AttributeTok{x =}\NormalTok{ Sample, }\AttributeTok{y =}\NormalTok{ log2\_norm\_rpkm, }\AttributeTok{fill =}\NormalTok{ Condition)) }\SpecialCharTok{+}
  \FunctionTok{geom\_boxplot}\NormalTok{() }\SpecialCharTok{+}
  \FunctionTok{theme}\NormalTok{(}\AttributeTok{axis.text.x =} \FunctionTok{element\_text}\NormalTok{(}\AttributeTok{angle =} \DecValTok{45}\NormalTok{, }\AttributeTok{hjust =} \DecValTok{1}\NormalTok{)) }\SpecialCharTok{+}
  \FunctionTok{labs}\NormalTok{(}\AttributeTok{title =} \StringTok{"Boxplots of Normalized RPKM by Sample and Condition"}\NormalTok{, }
       \AttributeTok{x =} \StringTok{"Sample"}\NormalTok{, }\AttributeTok{y =} \StringTok{"log2(Normalized RPKM)"}\NormalTok{)}
\FunctionTok{print}\NormalTok{(plot)}
\end{Highlighting}
\end{Shaded}

\includegraphics{group_normalised_data_files/figure-pdf/unnamed-chunk-12-1.pdf}

Saving the plot as an image

\begin{Shaded}
\begin{Highlighting}[]
\FunctionTok{ggsave}\NormalTok{(}\StringTok{"../Output/plots/boxplot\_rpkm\_comparison.jpg"}\NormalTok{, }
       \AttributeTok{plot =}\NormalTok{ plot, }\AttributeTok{width =} \DecValTok{10}\NormalTok{, }\AttributeTok{height =} \DecValTok{6}\NormalTok{, }\AttributeTok{dpi =} \DecValTok{300}\NormalTok{)}
\end{Highlighting}
\end{Shaded}





\end{document}
